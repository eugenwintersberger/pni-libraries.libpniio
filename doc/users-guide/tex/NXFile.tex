\nxfile\ is the root to create all other data structures. 
\nxfile\ is a descendant of \nxgroup\ and thus provides all the functionality of
its base class. Here we will only discuss features unique to the \nxfile\ class.
The next chapter deals with \nxgroup.
There is not too much specific with a file you can do except create or open one.
The former procedure is shown in this example:
%%\lstinputlisting{../examples/c++/nxfile_ex1.cpp}
\inputminted[linenos=true]{c++}{../examples/c++/nxfile_ex1.cpp}

The file is created in line $9$. Unlike other objects a file is not 
created by a constructor but by static factory methods of \nxfile.
The method {\tt NXFile::create\_file} takes three arguments: 
\begin{enumerate}
    \item the name of the file to create (here {\tt file\_ex1.h5})
    \item the overwrite flag which causes an already existing file of same name
    to be overwriten (as shown in the above example)
    \item the split size (set this to $0$ for now).
\end{enumerate}
To open a file use the factory method {\tt NXFile::open\_file} like this
\begin{minted}{c++}
    NXFile file = NXFile::open_file("file_ex1.h5",true);
\end{minted}
The only two arguments are
\begin{enumerate}
    \item the name of the file
    \item the read-only flag which causes the file to be opened in read-only
    mode of true.
\end{enumerate}


