%%%NXField documentation

Objects of type \nxfield\ are the data-holding entities in the world of \pninx.
As descendants of \nxobject\ they can hold attribute data as \nxgroup. 
Instances of \nxfield\ are created using the {\tt create\_field<T>} template
method provided by \nxfile\ and \nxgroup. 
This method comes in several version which will be briefly discussed here before
we continue with examples. 
Let us assume that  {\tt g} represents an instance of \nxgroup. The simplest 
approach to create a field is
\begin{minted}{c++}
    NXField field = g.create_field<Float32>("data");
\end{minted}
which will create a field with name {\tt data} below group {\tt g}. The rank
(number of dimensions) of the field is $1$ and a single data value can be stored
after creation. If a you want to create a multidimensional field an additional
shape argument must be supplied to the method like this
\begin{minted}{c++}
    Shape s = {1024,1024};
    NXField field = g.create_field<UInt16>("detector",s);
\end{minted}


\subsection{Fields with numerical data}
\label{section:nxfield_numeric_io}

\inputminted[linenos=true]{c++}{../examples/c++/nxnumfield_ex1.cpp}
\inputminted[linenos=true]{c++}{../examples/c++/nxnumfield_ex2.cpp}

\subsection{Fields with string data}
\label{section:nxfield_string_io}

\inputminted[linenos=true]{c++}{../examples/c++/nxstrfield_ex2.cpp}

\subsection{Fields with binary data}
\label{section:nxfield_binary_io}
The {\tt Binary} data-type provided by \pniutils\ has only one intention: hold
uninterpreted binary data. There are not too many applications for such data.
The most natural one might be to store a file in a Nexus file and latter write
it back to disk. This file might be an image file, a PDF document, or an office
file. In all cases the only operations applied to the binary data is reading and
writing. 
The next section presents such an application where an image file is read from
disk and stored to an \nxfield\ from which it will be read back and stored on
disk.
The main program of this program is shown below and looks rather trivial.
\inputminted[linenos,firstline=1,lastline=24]{c++}{../examples/c++/nxbinfield_ex1.cpp}
The scalar \nxfield\ instance holding the binary data is created in line 19.

\inputminted[linenos,firstnumber=27,firstline=27,lastline=44]{c++}{../examples/c++/nxbinfield_ex1.cpp}
\inputminted[linenos,firstnumber=47,firstline=47,lastline=56]{c++}{../examples/c++/nxbinfield_ex1.cpp}
