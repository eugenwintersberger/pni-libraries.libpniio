%%%NXField documentation

In the Nexus world the payload (your real data) is stored in fields which 
are represented in this library by objects of class {\tt NXField}.
Like groups, fields cannot be instantiated by their constructor but are 
rather created using factory methods provided by all classes derived 
from  {\tt NXGroup}. One word of caution: a prerequisit for understanding 
this section is a fundamental knowledge of the data objects provided 
by {\tt libpniutils}. If you are not please consult the {\tt libpniutils}
users guide first.
One can distinguish between two basic types of fields: string-fields
and numeric-fields. The former hold string data while the later 
holds numbers.   


\subsection{Fields with numerical data}
\label{section:nxfield_numeric_io}

\inputminted[linenos=true]{c++}{../examples/c++/nxnumfield_ex1.cpp}
\inputminted[linenos=true]{c++}{../examples/c++/nxnumfield_ex2.cpp}

\subsection{Fields with string data}
\label{section:nxfield_string_io}

\inputminted[linenos=true]{c++}{../examples/c++/nxstrfield_ex2.cpp}

\subsection{Fields with binary data}
\label{section:nxfield_binary_io}

\inputminted[linenos,firstline=1,lastline=24]{c++}{../examples/c++/nxbinfield_ex1.cpp}
\inputminted[linenos,firstnumber=27,firstline=27,lastline=44]{c++}{../examples/c++/nxbinfield_ex1.cpp}
\inputminted[linenos,firstnumber=47,firstline=47,lastline=56]{c++}{../examples/c++/nxbinfield_ex1.cpp}
