%%%NXField documentation

In the Nexus world the payload (your real data) is stored in fields which 
are represented in this library by objects of class {\tt NXField}.
Like groups, fields cannot be instantiated by their constructor but are 
rather created using factory methods provided by all classes derived 
from  {\tt NXGroup}. One word of caution: a prerequisit for understanding 
this section is a fundamental knowledge of the data objects provided 
by {\tt libpniutils}. If you are not please consult the {\tt libpniutils}
users guide first.
One can distinguish between two basic types of fields: string-fields
and numeric-fields. The former hold string data while the later 
holds numbers.   

\subsection{Creating fields}

The first step to use fields is to create them. Field creation for 
string and numeric fields is shown in this next example. 
\lstinputlisting{../examples/c++/nxfield_ex1.cpp}.
As we have seen in section~\ref{section:nxfield_design} the handling 
of data in \nxfield\ depends on the class of data that should be stored 




\subsection{Reading and writing numericaldata}
\label{section:nxfield_numeric_io}

\subsection{Reading and writing string data}
\label{section:nxfield_string_io}

\subsection{Reading and writing binary data}
\label{section:nxfield_binary_io}
