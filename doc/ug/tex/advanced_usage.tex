%%%===========================================================================
\section{The mysterious \nxobject}\label{section:nxobject}

In section~\ref{chapter:basicusage} we have discussed the basic usage of
\libpniio\ and the careful reader my already observed some missing concepts
\begin{itemize}
\item what is the mysterious return type of the \cpp{parent()} member function
every field, group, or attribute possesses?
\item if \cpp{nxgroup} is like a container -- can we use iteration with it?
\end{itemize}
Indeed solving the first riddle leads to a good solution for the iteration
problem as both are coupled by the fact that \cpp{nxgroup} is not just a simple
container type as the STL containers, it is rather a heterogeneous container. 

%%%---------------------------------------------------------------------------
\begin{figure}
    \centering
    \begin{tikzpicture}
        \node[classnode] (nxobject) {\cpp{nxobject}};
        \node[classnode,below left = of nxobject] (nxfield) {\cpp{nxfield}};
        \node[classnode,below right = of nxobject] (nxgroup) {\cpp{nxgroup}};


        \path[draw] (nxfield.north) |- ++(0cm,0.5cm)-|(nxobject.south);
        \path[draw,->] (nxgroup.north) |- ++(0cm,0.5cm)-|(nxobject.south);
    \end{tikzpicture}
    \caption{{\small\label{fig:advanced:hierarchy}
    The default procedure in OOP to obtain a mast object which can either represent a
    field or a group would be inheritance. In this schema \nxfield\ and
    \nxgroup\ are both descendants of \nxobject. 
    }}
\end{figure}
%%%---------------------------------------------------------------------------

But lets start at the beginning -- the return type of \cpp{parent()}. If we
would have only fields and groups the situation would be easy: the return type
of the \cpp{parent()} function for a field or group would always be
\cpp{nxgroup} (as fields cannot have children for their own). 
However, what about attributes. They can be attached to fields or groups. Thus,
also the return type would be either a field or a group. Unfortunately, a
function (and thus a method too) can only have one return type. We would need a
return type that could represent either a field or a group. 
The classical approach to solve such an issue in object oriented programming
would be a class hierarchy as depicted in Fig.~\ref{fig:advanced:hierarchy}.

However, this approach has some flaws. What interface should be base class
expose. In fact, the parts of their interface \nxfield\ and \nxgroup have in
common also matches those of \nxattribute\ too. However, all three classes are
from a semantic point of view to different to be derived from a single base
class.  There is also a less philosophical problem. \nxfield\ and \nxgroup\ are
in fact templates whose template parameter is the implementation type (the same
is true for \nxattribute). The basic idea was to implement a \emph{pointer to
implementation} pattern with template (so in fact no pointers) following an
approach presented by \cite{vandevoorde:2002}. This approach has severe
consequences for a base class approach: as the base class would need access to
the particular implementation and thus the implementation type, a class 
hierarchy must be assembled for each implementation type. During the 
development of \libpniio\ such an approach was once used and has proven to be 
far to complex. 

To circumvent all this issues and in general avoid design by inheritance (as
suggested by \cite{gof:1995}) a different approach was taken. You may have
already recognized in the API documentation an instance of type \nxobject. 
\nxobject\ mimics a bit the behavior of a base class but in reality it is
something entirely different. It is an instance of a \emph{variant} template 
provided by the \cpp{boost::variant} library.
\nxobject\ thus can hold either an instance of \nxgroup, \nxfield, or
\nxattribute. It is thus the perfect return type for the \cpp{parent()} 
member function of field, groups, and attributes. Indeed \nxobject\ is also what
the \cpp{[]} operators and the \cpp{at()} member function of \nxgroup\ returns. 
By the special design of the constructor as well as the assignment operator
of \nxfield, \nxgroup, and \nxattribute, these objects can be directly
constructed from an instance of \nxobject\ if the instance of \nxobject\ holds
an instance of the appropriate type. 
Thus, \nxobject\ is rather transparent to the user. 
In the next sections \ref{section:algorithms} and \ref{section:group_iteration}
we will see how to work with instance of \nxobject\ effectively.

It is important to understand that \nxobject\ depends on the particular
implementation in use. Thus it is part of the particular implementation
namespace. As \libpniio\ currently only uses HDF5 the appropriate type would be 
\cpp{h5::nxobject}.



%%%===========================================================================
\section{Using algorithms}\label{section:algorithms}

\nxobject\ does not expose any interface functions. Thus, all operations on 
instances of \nxobject\ are implemented as algorithms using the visitor pattern. 
Algorithms can be used by adding 
\begin{cppcode}
#include <pni/io/nx/algorithms.hpp>
\end{cppcode}
at the top of a source file. All the function templates provided 
reside with in the \cpp{pni::io::nx} namespace. 
Some of the algorithms also work with the standard types like \nxfield,
\nxgroup, and \nxattribute. Which one will be explained in the following
section.

\subsection{Basic inquiry and conversion}

When accessing the elements of a group or when retrieving the parent object of
an attribute the first question one may raises is: what kind of object is stored
in the \nxobject\ instance. 
There are basically three simple functions: \cpp{is\_attribute()},
\cpp{is\_field()}, and \cpp{is\_group()}. Each of these functions takes as their
single argument an instance of \nxobject. 
\begin{cppcode}
h5::nxgroup g = file.root();

for(size_t i=0;i<g.size();++i)
{
    auto object = g[i]; //this returns an instance of nxobject 

    if(is_field(object))
        std::cout<<"found a field"<<std::endl;
    else if(is_group(object))
        std::cout<<"found a group"<<std::endl;
    else
        std::cout<<"found unknown object type!"<<std::endl;
}
\end{cppcode}
The first line in this example, \cpp{h5::nxgroup g = file.root();} is a 
good example for implicit conversion. In fact \cpp{root()} returns an instance
of \nxobject\ rather than of \nxgroup. However, \nxgroup\ can be constructed 
from \nxobject\ as long as \nxobject\ stores an instance of \nxgroup.

There are also three conversion function templates available: \cpp{as\_field()}, 
\cpp{as\_group()}, and \cpp{as\_attribute()} each taking an instance of 
\nxobject\ and returning a field, group, or attribute instance. If the 
\nxobject\ argument does not store a value of appropriate type a
\cpp{type\_error} exception will be thrown. 
We could modify the \cpp{if} block in the above example with 
\begin{cppcode}
if(is_field(object))
    do_something_with_field(as_field(object));
else if(is_group(object))
    do_something_else_with_groups(as_group(object));
else
    std::cout<<"found unknown object type!"<<std::endl;
\end{cppcode}
None of these algorithms will work on either \nxfield, \nxgroup, or
\nxattribute. It would not make too much sense to ask an instance of \nxfield\
whether or not it is a field. We know this already. 

\subsection{Common algorithms for fields, groups, and attributes}

Fields, groups, and attributes share a set of common member functions which can
be accessed by a set of algorithms acting on an instance of \nxobject\ which 
stores either a field, a group, or an attribute. 

\subsubsection{\cppul{is\_valid()}}

One of the most important questions to ask from an object is whether or not it
is valid. The \cpp{is\_valid()} algorithms does exactly that. It behaves like 
the \cpp{is\_valid()} member function of \nxgroup, \nxfield, or \nxattribute. 
Indeed, this algorithms does not only work for \nxobject, \nxgroup, \nxfield,
and \nxattribute\ but also for \nxfile. Using it is fairly simple 
\begin{cppcode}
auto object = .....; 

if(!is_valid(object))
    std::cerr<<"Object not valid!"<<std::endl;
\end{cppcode}
The return value is of type \cpp{bool} and has the value \cpp{true} when the
object is valid, \cpp{false} otherwise.

\subsubsection{\cppul{close()}}

This template function works for \nxgroup, \nxfield, \nxattribute, and
\nxobject. It does what the name suggests: it calls the \cpp{close()} method of
the object passed as its only argument. 
\begin{cppcode}
auto object = ...; //can be either a field,group, attribute, or object

close(object);
\end{cppcode}
This function template has no return value.

\subsubsection{\cppul{get\_filename()}}

Calls the \cpp{get\_filename()} method on the object passed as an argument. The
function template works with \nxobject, \nxfield, \nxgroup, and \nxattribute. 
\begin{cppcode}
auto object = ...; //can be either a field,group, attribute, or object

std::cout<<"File is: "<<get_filename(object)<<std::endl;
\end{cppcode}
The return value is a string with the name of the file.

\subsubsection{\cppul{get\_root()}}

Returns the root group of the file an object belongs to. I accepts \nxobject,
\nxattribute, \nxfield, and \nxgroup\ instances as arguments. 
\begin{cppcode}
auto object = ...; //can be either a field,group, attribute, or object

h5::nxgroup root = get_root(object);
\end{cppcode}
The root group is returned as an instance of \nxobject. As shown in the short
code snippet above we do not have to care about this as \nxgroup\ can be
constructed from an \nxobject.

\subsubsection{\cppul{get\_name()}}

Returns the name of an object. The function templates takes an instance of 
\nxobject, \nxfield, \nxattribute, \nxgroup\ as its only argument
\begin{cppcode}
auto object = ...; //can be either a field,group, attribute, or object

std::cout<<"Name: "<<get_name(object)<<std::endl;
\end{cppcode}
The name is returned as a string. If the object is not valid an
\cpp{invalid\_object\_error} exception is thrown.

\subsubsection{\cppul{get\_parent()}}

Return the parent of an object. Takes an instance of \nxobject, \nxfield,
\nxattribute, and \nxgroup\ as its only argument
\begin{cppcode}
auto object = ...; //can be either a field,group, attribute, or object
auto p = get_parent(object);
\end{cppcode}
The parent object is returned as an instance of \nxobject. If the object is not
valid an \cpp{invalid\_object\_error} is thrown.

\subsubsection{\cppul{get\_path()}}

The \cpp{get\_path} function template takes either an \nxobject, \nxattribute,
\nxfield, and \nxgroup\ as its single argument.
\begin{cppcode}
auto object = ...; //can be either a field,group, attribute, or object
std::cout<<get_path(object)<<std::endl;
\end{cppcode}
The path is returned as a string. This is maybe not seem to be the best approach
but strings are most probably more often printed than used for iteration. 
However is a path object is required one can always do
\begin{cppcode}
nxpath p = nxpath::from_string(get_path(object));
\end{cppcode}

\subsubsection{\cppul{get\_size()}}

Returns the size of an object. This function template accepts an instance of 
\nxobject, \nxattribute, \nxfield, or \nxgroup\ as its single input argument.
\begin{cppcode}
auto object = ...; //can be either a field,group, attribute, or object
size_t n = get_size(object);
\end{cppcode}
The interpretation is slightly different whether or not the argument 
represents a group, a field, or an attribute. 
\begin{itemize}
\item for fields and attributes the return value is the total number of 
elements stored in the field or attribute 
\item for groups it is the total number of children.
\end{itemize}



\subsection{Group related algorithms}

The set of algorithms described in this section make only sense in connection 
with a group object.

%%%===========================================================================
\subsubsection{\cppul{create\_field()}}

The \cpp{create\_field} function template works only with an instance of
\nxobject\ which stores an instance of \nxgroup. 
In the simplest case the \cpp{create\_field} 
\begin{cppcode}
auto detector = instrument["detector"];

auto field = create_field<float64>(detector,"../temperature");
\end{cppcode}
The template parameter of the \cpp{create\_field} function template, in this
example, determines the data type which should be used for the field. 
The first argument is the parent object for the field. In this example it is the 
group detector represented by an instance of \nxobject. The second argument is
the path to the new field. In this case this particular case the new field will
not be created no below the \emph{detector} but below the \emph{instrument}. In
fact, the parent object (the first argument) is just the starting point for the
path to the new field. The path must exist except for the last element (which is
the name of the newly created field). 
The path can either be passed to the function template as a string or as an
instance of \nxpath. 

Multidimensional fields can be created by appending the shape of the field 
to the list of arguments of the \cpp{create\_field} function template
\begin{cppcode}
auto detector = instrument["detector"];

auto field = create_field<float64>(detector,"background",shape_t{1024,1024});
\end{cppcode}
A filter can simply be added by appending it to the argument list
\begin{cppcode}
h5::nxdeflate_filter comp = ...;
auto detector = instrument["detector"];
shape_t shape{1024,1024};

auto field = create_field<float64>(detector,"background",shape,comp);
\end{cppcode}
A description of custom chunking can be found in
section~\ref{section:field_chunks}.
One of the disadvantages of these version of the \cpp{create\_field} function
template is the fact, that the data type can only be determined at compile time. 
To overcome this problem there is also a runtime version 
\begin{cppcode}
h5::nxdeflate_filter comp = ...;
auto detector = instrument["detector"];
shape_t shape{1024,1024};

auto field = create_field(detector,type_id::FLOAT64,"background",shape,comp);
\end{cppcode}
where the second argument is the type ID of the desired data type. All other
arguments are the same as for the previous versions of \cpp{create\_field}.

%%%===========================================================================
\subsubsection{\cppul{create\_group()}}

Creating groups using algorithms is much more powerful than the standard
\cpp{create\_group} method of \nxgroup. 
The \cpp{create\_group} algorithm requires two arguments where the first one is
the parent object for the new group and the second one the path to the new 
group. The \cpp{creat\_group} algorithm uses the \nexus-path syntax to determine
the type of the newly created group.
In it simplest form, in order to achieve the same result as with the 
\cpp{create\_group} member function, one could use
\begin{cppcode}
auto root = ...;
auto entry = create_group(root,"entry:NXentry");
\end{cppcode}
which would create a new group of name \cpp{entry} and of type \cpp{NXentry}.
A more elaborate example would be 
\begin{cppcode}
auto detector = ...
auto mono     = create_group(detector,"../monochromator:NXmonochromator");
\end{cppcode}
Instead of its string representation, an instance of \nxpath\ can be used 
to determined position, name, and class of the new group.



%%%===========================================================================
\subsubsection{\cppul{get\_child()}}

Retrieves the child of a group object. The child can be identified either by its
index or by its name (and class in case of a group). 
\begin{cppcode}
auto entry = ....;

for(size_t i=0;i<get_size(entry);++i) 
{
    auto child = get_child(entry,i);

    //do something with the child
}
\end{cppcode}
The parent can either be an instance of \nxgroup\ or \nxobject.  A
\cpp{type\_error} exception is thrown in the case that an \nxobject-parent does
not store an instance of \nxgroup.
More useful, however, is to search for a particular object by its type
\begin{cppcode}
auto instrument = ...;
auto detector   = get_child(instrument,"","NXdetector");
\end{cppcode}
or by its name
\begin{cppcode}
auto instrument = ...;
auto monitor    = get_child(instrument,"monitor");
\end{cppcode}
or even both
\begin{cppcode}
auto instrument = ....;
auto detector_1 = get_child(instrument,"det_1","NXdetector");
\end{cppcode}
In all, of the last three examples, if there are more than one instance in the 
parent which would satisfy the requirement, the first one will be returned. 

%%%===========================================================================
\subsubsection{\cppul{get\_children()}}
\todo[inline]{To be implemented}


%%%===========================================================================
\subsubsection{\cppul{get\_class()}}

The \cpp{get\_class} function template works for instances of \nxgroup\ and
\nxobject. It returns the value of the \cpp{NX\_class} attribute which 
determines the base class type of the group.
\begin{cppcode}
if(get_class(group) == "NXentry")
    //do something with the entry
else
    throw type_error(EXCEPTION_RECORD,"group is not an entry!");
\end{cppcode}
If the attribute is not set or does not exist the function returns an empty
string. 

%%%===========================================================================
\subsubsection{\cppul{get\_object()}}

Aside from the \cpp{get\_child} function template, this is most probably one of
the most important function templates presented in this section. 
The common methods to access children of a group usually only allow you to 
access the direct children of a group. The \cpp{get\_child} function template 
allows us to go beyond this limitation. 
We can use a path relative to a parent object
\begin{cppcode}
auto detector = ....;
auto monitor  = get_object(detector,"../:NXmonitor");
\end{cppcode}
Or even an absolute path
\begin{cppcode}
auto detector = ....;
auto monitor  = get_object(detector,"/:NXentry/:NXmonitor");
\end{cppcode}
Quite similar to \cpp{get\_child}, if several objects exist that would match
the path, the first one will be returned. The path to the object can be passed 
either as a string or as an instance of \nxpath.

%%%===========================================================================
\subsubsection{\cppul{get\_objects()}}
\todo[inline]{To be implemented!}

%%%===========================================================================
\subsubsection{\cppul{is\_class()}}

The \cpp{is\_class} function template checks if a group is an instance of a
particular base class. 
\begin{cppcode}
if(is_class(group,"NXentry"))
    //process NXentry
else
    throw type_error(EXCEPTION_RECORD,"group is not an instance of NXentry!");
\end{cppcode}
This function works for instances of \nxgroup\ as well as for \nxobject.  If
\nxobject\ does not hold an instance of \nxgroup\ a \cpp{type\_error} exception
will be thrown.

%%%===========================================================================
\subsubsection{\cppul{set\_class()}}

Set the base class type for a particular group. 
\begin{cppcode}
set_class(group,"NXentry");
\end{cppcode}
The first argument is the group for which to set the type. It must be either an
instance of \nxgroup\ or \nxobject. In the latter case a \cpp{type\_error}
exception is thrown if the \nxobject\ does not store a group instance.

If the group has already its \cpp{NX\_class} attribute set, the old value will be
overwritten.


%%%===========================================================================
\section{Iterating groups}\label{section:group_iteration}

\subsection{Simple iteration}

The {\tt nxgroup} type provides an STL compliant iterator interface to iterate
over the direct children of a group. In this very simple example we loop over 
all entries stored in a file
\begin{cppcode}
h5::nxgroup root = f.root();

for(auto entry: root)
    std::cout<<entry.name()<<std::endl;
\end{cppcode}
Another interesting example would be to count all instances of {\tt NXdetector} 
within an instrument group
\begin{cppcode}
#include <pni/core/types.hpp>
#include <pni/io/nx/nx.hpp>
#include <pni/io/nx/algorithms.hpp>

using namespace pni::core;
using namespace pni::io::nx;

//predicate function
bool is_detector(const h5::nxobject &o)
{
    if(is_group(o)) return is_class(o,"NXdetector");
    else return false;
}

int main(int argc,char **argv)
{
    h5::nxfile file = h5::nxfile::open_file("test.nxs");
    h5::nxgroup instrument = get_object(root,"/:NXentry/:NXinstrument");

    size_t ndetectors = std::count_if(instrument.begin(),instrument.end(),
                                      is_detector);
    std::cout<<"Found "<<ndetectors<<" detectors!"<<std::endl;
    return 0;
}
\end{cppcode}

\subsection{Recursive iteration}

\subsection{Deleting items}


%%%===========================================================================
\section{Custom field chunks}\label{section:field_chunks}

Chunking is an important issue but requires some more detailed information about
how the data is stored. That's why this section concludes this advanced usage
chapter. Chunking is particularly important for multidimensional data. 
The chunking feature is unique to the HDF5 library und thus maybe not available 
in future other implementations.

\subsection{What are chunks?}

%%%---------------------------------------------------------------------------
\begin{figure}[tb]
    \centering
    \begin{tikzpicture}
        \node (data) {
        \begin{tikzpicture}
            \draw[thick,step=0.5cm] (0,0) grid (10cm,3cm);
        \end{tikzpicture}
        };

        \node[above left = -0cm and -0.25cm of data] (origin) {(0,0)};
        \draw[->] (origin.south) -- node[below=0.5cm] {i} +(0,-1cm);
        \draw[->] (origin.east) -- node[right=0.5cm] {j} +(1cm,0);

        \begin{scope}[on background layer]
        \node[rectangle,rounded corners,draw,fit=(origin)(data)](frame){};
        \end{scope}
    \end{tikzpicture}
    \caption{{\small\label{fig:no_chunks}
    Nexus field without chunks: $i$ denotes the slow and $j$ the fast dimension.
    The entire field has to be written as once.
    }}
\end{figure}
%%%---------------------------------------------------------------------------
Without chunking a field is always accessed as a whole. This can be severe issue
if the field is very large (maybe even larger than the available main memory).
Figure~\ref{fig:no_chunks} shows such a situation. 
Consequently, whenever data is accessed the entire field has to be read or
written from and to the disk.
In order to avoid this situation data can be subdivided in to smaller
\emph{chunks}. Assuming a multidimensional field, chunks can be considered as
slices (selections) within this field. A single chunk is contiguously written to
disk. However, the next chunk might be written on an entirely different location
within the file rather than 
In order to avoid this situation data is written in smaller portions: the so
called \emph{chunks}. A chunk can be considered a slice of the multidimensional
field which should be written. This slice represents the unit of data which 
is contiguously written to disk. 
%%%---------------------------------------------------------------------------
\begin{figure}[tb]
    \centering
    \begin{tikzpicture}
        \node[outer sep=0cm,node distance=0cm] (cl1) 
        { \tikz\draw[thick,step=0.25cm] (0,0) grid +(5cm,0.25cm);};

        \foreach \co / \ca in {1/2,2/3,3/4,4/5}
            \node[below = -0.28cm of cl\co,outer sep=0cm,node distance=0cm]
            (cl\ca) 
            { \tikz\draw[thick,step=0.25cm] (0,0) grid +(5cm,0.25cm);};

        %%%---------------draw the chunks--------------------------------
        \node[above right = 0cm and 1cm of cl1] (chunk1) {
        \tikz \draw[thick,step=0.25cm] (0,0) grid +(5cm,0.25cm);
        };

        \foreach \co / \ca in {1/2,2/3,3/4,4/5}
            \node[below = 0cm of chunk\co] (chunk\ca) {
            \tikz \draw[thick,step=0.25cm] (0,0) grid +(5cm,0.25cm);
            };

        %%%-----------------draw connections between chunks and slices--------
        \foreach \cn in {1,2,3,4,5}
            \draw[thick,->] (cl\cn.east) -- (chunk\cn.west);
       
        %%%---------------draw chunk index------------------------------------
        \foreach \cn in {1,2,3,4,5}
            \node[right = 0cm of chunk\cn,red] {\cn};

        \node[above=of cl1] {field without chunks};
        \node[above=of chunk1] {field with chunks};

    \end{tikzpicture}
    \caption{{\small\label{fig:with_chunks}
    The field from Fig.~\ref{fig:no_chunks} is distributed over several chunks. 
    In this case the chunks are chosen along the fast dimension of the field 
    which is usually a good choice (assuming C-ordering of multidimensional
    data). The red numbers indicate the chunk index. The individual chunks do
    not have to be stored consecutively  in memory.
    }}
\end{figure}
%%%---------------------------------------------------------------------------
Figure~\ref{fig:with_chunks} shows an example how a \nexus\ field is distributed
over several chunks. In this example the chunks have been chosen along the 
fast dimension. This is typically a good choice if multidimensional data is
stored in C-order. 
Chunking is entirely transparent to the user. The way how fields are distributed
in chunks does not influence the read and write functions. However, it can have
significant influence on the performance. This is particularly true when using
compression. Without chunks the entire field is compressed. This can cause
dramatic performance penalties when accessing data as the entire data must be
read from disk and afterwards been inflated. When using chunks the compression
is only applied to a single chunk. Though this may slightly reduced the
compression ration which can be achieved, it can given an extreme performance
boost when reading data. Only the chunk where the requested data resides in must
be read from disk and being inflated. 

\subsection{Setting the chunk shape}

The size of a chunk must be set during field creation. Usually you do not have
to care about chunking. By default \libpniio\ sets the chunks shape to a
reasonable value. However, in some special cases you may want to use a 
custom chunk shape. 
This can be done with the \nxgroup\ member function templates used to create
field, and with the corresponding algorithm function. 
For the former one you can use
\begin{cppcode}
h5::nxgroup detector = get_object(root,":NXentry/:NXinstrument/:NXdetector");

shape_t shape = {0,1024,1024};
shape_t chunk = {1,1024,1024};

h5::nxfield data = detector.create_field<uint32>("data",shape,chunk);
\end{cppcode}
Here we start with shape where the first dimension is $0$. The idea is to grow
the field along this dimension with every detector frame which should be stored. 
The chunk shape, however, must not be of size $0$ (it would also not make
sense). We chose here a chunk shape which covers a single frame. For 2D
detectors this is a quite reasonable choice. 
This can also be used along with a compression filter
\begin{cppcode}
h5::nxdeflate_filter comp(2,true);
h5::nxgroup detector = get_object(root,":NXentry/:NXinstrument/:NXdetector");

shape_t shape = {0,1024,1024};
shape_t chunk = {1,1024,1024};

h5::nxfield data = detector.create_field<uint32>("data",shape,chunk,comp);
\end{cppcode}
The same is true for algorithms version of \cpp{create\_field}. The last line of
the above snippet would change to
\begin{cppcode}
h5::nxfield data = create_field(h5::nxobject(detector),"data",shape,chunk,comp);
\end{cppcode}

