Chunking is an important issue but requires some more detailed information about
how the data is stored. That's why this section concludes this advanced usage
chapter. Chunking is particularly important for multidimensional data. 
The chunking feature is unique to the HDF5 library und thus maybe not available 
in future other implementations.

\subsection{What are chunks?}

%%%---------------------------------------------------------------------------
\begin{figure}[tb]
    \centering
    \begin{tikzpicture}
        \node (data) {
        \begin{tikzpicture}
            \draw[thick,step=0.5cm] (0,0) grid (10cm,3cm);
        \end{tikzpicture}
        };

        \node[above left = -0cm and -0.25cm of data] (origin) {(0,0)};
        \draw[->] (origin.south) -- node[below=0.5cm] {i} +(0,-1cm);
        \draw[->] (origin.east) -- node[right=0.5cm] {j} +(1cm,0);

        \begin{scope}[on background layer]
        \node[rectangle,rounded corners,draw,fit=(origin)(data)](frame){};
        \end{scope}
    \end{tikzpicture}
    \caption{{\small\label{fig:no_chunks}
    Nexus field without chunks: $i$ denotes the slow and $j$ the fast dimension.
    The entire field has to be written as once.
    }}
\end{figure}
%%%---------------------------------------------------------------------------
Without chunking a field is always accessed as a whole. This can be severe issue
if the field is very large (maybe even larger than the available main memory).
Figure~\ref{fig:no_chunks} shows such a situation. 
Consequently, whenever data is accessed the entire field has to be read or
written from and to the disk.
In order to avoid this situation data can be subdivided in to smaller
\emph{chunks}. Assuming a multidimensional field, chunks can be considered as
slices (selections) within this field. A single chunk is contiguously written to
disk. However, the next chunk might be written on an entirely different location
within the file rather than 
In order to avoid this situation data is written in smaller portions: the so
called \emph{chunks}. A chunk can be considered a slice of the multidimensional
field which should be written. This slice represents the unit of data which 
is contiguously written to disk. 
%%%---------------------------------------------------------------------------
\begin{figure}[tb]
    \centering
    \begin{tikzpicture}
        \node[outer sep=0cm,node distance=0cm] (cl1) 
        { \tikz\draw[thick,step=0.25cm] (0,0) grid +(5cm,0.25cm);};

        \foreach \co / \ca in {1/2,2/3,3/4,4/5}
            \node[below = -0.28cm of cl\co,outer sep=0cm,node distance=0cm]
            (cl\ca) 
            { \tikz\draw[thick,step=0.25cm] (0,0) grid +(5cm,0.25cm);};

        %%%---------------draw the chunks--------------------------------
        \node[above right = 0cm and 1cm of cl1] (chunk1) {
        \tikz \draw[thick,step=0.25cm] (0,0) grid +(5cm,0.25cm);
        };

        \foreach \co / \ca in {1/2,2/3,3/4,4/5}
            \node[below = 0cm of chunk\co] (chunk\ca) {
            \tikz \draw[thick,step=0.25cm] (0,0) grid +(5cm,0.25cm);
            };

        %%%-----------------draw connections between chunks and slices--------
        \foreach \cn in {1,2,3,4,5}
            \draw[thick,->] (cl\cn.east) -- (chunk\cn.west);
       
        %%%---------------draw chunk index------------------------------------
        \foreach \cn in {1,2,3,4,5}
            \node[right = 0cm of chunk\cn,red] {\cn};

        \node[above=of cl1] {field without chunks};
        \node[above=of chunk1] {field with chunks};

    \end{tikzpicture}
    \caption{{\small\label{fig:with_chunks}
    The field from Fig.~\ref{fig:no_chunks} is distributed over several chunks. 
    In this case the chunks are chosen along the fast dimension of the field 
    which is usually a good choice (assuming C-ordering of multidimensional
    data). The red numbers indicate the chunk index. The individual chunks do
    not have to be stored consecutively  in memory.
    }}
\end{figure}
%%%---------------------------------------------------------------------------
Figure~\ref{fig:with_chunks} shows an example how a \nexus\ field is distributed
over several chunks. In this example the chunks have been chosen along the 
fast dimension. This is typically a good choice if multidimensional data is
stored in C-order. 
Chunking is entirely transparent to the user. The way how fields are distributed
in chunks does not influence the read and write functions. However, it can have
significant influence on the performance. This is particularly true when using
compression. Without chunks the entire field is compressed. This can cause
dramatic performance penalties when accessing data as the entire data must be
read from disk and afterwards been inflated. When using chunks the compression
is only applied to a single chunk. Though this may slightly reduced the
compression ration which can be achieved, it can given an extreme performance
boost when reading data. Only the chunk where the requested data resides in must
be read from disk and being inflated. 

\subsection{Setting the chunk shape}

The size of a chunk must be set during field creation. Usually you do not have
to care about chunking. By default \libpniio\ sets the chunks shape to a
reasonable value. However, in some special cases you may want to use a 
custom chunk shape. 
This can be done with the \nxgroup\ member function templates used to create
field, and with the corresponding algorithm function. 
For the former one you can use
\begin{cppcode}
h5::nxgroup detector = get_object(root,":NXentry/:NXinstrument/:NXdetector");

shape_t shape = {0,1024,1024};
shape_t chunk = {1,1024,1024};

h5::nxfield data = detector.create_field<uint32>("data",shape,chunk);
\end{cppcode}
Here we start with shape where the first dimension is $0$. The idea is to grow
the field along this dimension with every detector frame which should be stored. 
The chunk shape, however, must not be of size $0$ (it would also not make
sense). We chose here a chunk shape which covers a single frame. For 2D
detectors this is a quite reasonable choice. 
This can also be used along with a compression filter
\begin{cppcode}
h5::nxdeflate_filter comp(2,true);
h5::nxgroup detector = get_object(root,":NXentry/:NXinstrument/:NXdetector");

shape_t shape = {0,1024,1024};
shape_t chunk = {1,1024,1024};

h5::nxfield data = detector.create_field<uint32>("data",shape,chunk,comp);
\end{cppcode}
The same is true for algorithms version of \cpp{create\_field}. The last line of
the above snippet would change to
\begin{cppcode}
h5::nxfield data = create_field(h5::nxobject(detector),"data",shape,chunk,comp);
\end{cppcode}

