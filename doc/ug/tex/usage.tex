%%%describing the usage of the library

In this section we will have a short look on how to make your code working the
\libpniio. The key to make using \libpniio\ simple is the usage of {\tt
pkg-config}. 

\section{From the command line}

\begin{minted}{bash}
    $> g++ -std=c++11 -otest test.cpp $(pkg-config --cflags --libs pniio)
\end{minted}
There are two important remarks we have to make here. The first is the {\tt
-std=c++11} after {\tt g++}. This tells the compiler to use the new C++11
standard. This option is absolutely required for the code to build. 
the {\tt pkg-config} command at the end of the command line includes all the
necessary compiler and linker flags to build and link the code.

\section{From within a Makefile}

\section{With CMake}
