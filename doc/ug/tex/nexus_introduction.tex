%%% a basic introduction to Nexus

Today, data recorded during experiments is typically stored either as individual
image files or as a flat ASCII file. In some situations both are used where
image data from detectors is stored in a binary image format while scalar data
is stored in a flat ASCII file. 
This approach has several problems 
\begin{enumerate}
\item when the number of image files grows large the performance of most file
systems degenerate 
\item to access data in an individual image file a new file handler has to be
created 
\item image and scalar data is stored in different files which increases the
managements efforts to keep related information aligned.
\end{enumerate}

Nexus is a binary file format which attempts to solves all these problems. Nexus
can keep scalar and multidimensional data within a single file and allows to
organize data in trees. Additional attributes can be attached to each object in
a file to store metadata which might be required for later analysis of the data.
