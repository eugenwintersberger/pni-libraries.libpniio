
%%%---------------------------------------------------------------------------
\begin{figure}
\centering
\begin{tikzpicture}
    \node[nxgroup] (entry) {\nodepart{one} entry \nodepart{two} \nxentry};
    \node[nxgroup,right = of entry] (instrument) {\nodepart{one} instrument 
                               \nodepart{two} \nxinstrument};
    \node[nxgroup, right = of instrument] (detector) {\nodepart{one} detector
                             \nodepart{two} \nxdetector};
    \node[nxfieldh=5, right = of detector] (detectordata) 
                                        {\nodepart{one} data
                                         \nodepart{two} $1.2$ 
                                         \nodepart{three} $4.3$ 
                                         \nodepart{four} $7.89$ 
                                         \nodepart{five} \dots
                                         };
    \node[nxgroup, below = of instrument] (data) {\nodepart{one} data
                         \nodepart{two} \nxdata};
    \node[nxfieldh=2,right = of data] (datalink)
                                   {\nodepart{one} data
                                    \nodepart{two}
                                    \texttt{../instrument/detector/data}
                                   };

    \draw[thick,->] (entry.east) -- (instrument.west);
    \draw[thick,->] (instrument.east) -- (detector.west);
    \draw[thick,->] (detector.east) -- (detectordata.west);
    \draw[thick,->] ($(entry.east)!0.5!(instrument.west)$) |- (data.west);
    \draw[thick,->] (data.east)--(datalink.west);
    \draw[thick,dashed,->] (datalink.north) --
                           node[midway,right] {link}
                           (detectordata.south west);
\end{tikzpicture}
\caption{{\small\label{fig:link:internal_link}
Links can be used to make data available in two different group without
duplicating the data. Here the data stored in the detector is also available
below the data group in the entry group of the tree. 
}}
\end{figure}
%%%---------------------------------------------------------------------------

%%%---------------------------------------------------------------------------
\begin{figure}
    \centering
    \begin{tikzpicture}[level distance = 2cm]
        \node[nxgroup] (master entry) { entry \nodepart{two} \nxentry}
          [growth parent anchor = north east] 
          child { 
                  node[nxgroup] (master instrument) {instrument 
                                               \nodepart{two} \nxinstrument}
                  child {
                    node[nxgroup] (master detector) { detector
                                               \nodepart{two} \nxdetector}
                      child{
                        node[nxfieldv=2] (master link) { data \nodepart{two} 
                        \texttt{\tiny
                        detector.nxs://entry/instrument/detector/data} 
                        }
                      }
                  }
                };

         \node[nxgroup,right = 7cm of master entry] (detector entry) 
                 { entry \nodepart{two} \nxentry}
          [growth parent anchor = north east] 
          child { 
                  node[nxgroup] (detector instrument) {instrument 
                                               \nodepart{two} \nxinstrument}
                  child {
                    node[nxgroup] (detector detector) { detector
                                               \nodepart{two} \nxdetector}
                      child{
                        node[nxfieldv=2] (detector data) 
                        { data \nodepart{two} 
                          \begin{tikzpicture}
                          \draw[step=1mm] (0,0) grid (20mm,5mm); 
                          \end{tikzpicture}
                        }
                      }
                  }
                };
        
            \draw[thick,dashed,->] (master link.south) .. 
                                   controls ($(master link.south)!0.1!(detector
                                   data.south)+(0cm,-1.5cm)$) and 
                                   ($(master link.south)!0.9!(detector
                                   data.south)+(0cm,-1.5cm)$)
                                   .. node[midway,below]{external link}(detector data.south);

            \begin{scope}[on background layer]
                \node[rectangle, draw,rounded corners,fit=(master entry)(master
                link)] (master file){};
                \node[above = 0.25cm of master file,align=center] {\nexus\
                master file\\ \texttt{master.nxs}};
                \node[rectangle, draw,rounded corners,fit=(detector entry)
                (detector data)] (detector file) {};
                \node[above = 0.25cm of detector file,align=center] 
                {\nexus\ detector file\\ \texttt{detector.nxs}};
            \end{scope}
    \end{tikzpicture}
    \caption{{\small\label{fig:link:external_link}
    An external link used to reference the data stored in a separate detector
    file via the data field in the master file. The master file is what the user
    typically uses to access the data. 
    }}
\end{figure}
%%%---------------------------------------------------------------------------

One of the key features of \nexus\ is its support for links. Like on a file
system linking allows for making data available at different positions in the
\nexus\ group hierarchy without data duplication. 
A typical application for a link would be the data stored in \nxdata\ instance
of a \nexus-file. \nexus\ distinguishes two kinds of links
\begin{itemize}
\item internal links - where the link and its target are located in the same
file
\item and external links - where the target resides in a different file than the 
link.
\end{itemize}
For both kinds of links there are canonical use cases. 
The standard use case for internal links is depicted in
Fig.~\ref{fig:link:internal_link}. Here the link is used to reference the data
stored in the detector group of the file in its data group\footnote{This group
is used for programms to quickly find plottable data.}.
By using links we only have to store the data once (in the detector group) and
can make it available at any other position in the file.

The default use case for external links is shown in
Fig.~\ref{fig:link:external_link}. In many cases detector write to a network
storage while entirely bypassing the control system of the experiment. This is
typically done when the detector has to record images hat a very hight rate
which could not have been achieved if all the data has to be pumped through the
control systems software stack. 
In the best case, as shown in Fig.~\ref{fig:link:external_link}, the detector
already writes its data in a \nexus\ file which contains nothing else than the
detector images. While the detector is recording images the control system
writes all the metadata and other relevant data to a master file (this data is
usually rather small and not recorded at low frequency). As a result we end up
with two files and the user would have to access metadata through the master
file and the detector images via the detector file. By using external links 
it is possible to make data within a \nexus-file availabe that is originally
stored in a different \nexus-file. Now the user can access the detector data via
the master file as it would be stored directly within this file.
The only thing the user has to remember is that he or she needs to copy two
files when movin the data to a different storage.

%%%===========================================================================
\subsection{Create internal links}

To create an internal link the \cpp{pni::io::nx} namespace provides the
\cpp{link()} function template which exists in several overloaded versions.
All of them differ virtually only in the way how the link target is referenced
which is the first argument of \cpp{link()}. The second and third argument are
the parent group for the new link and its name.

The most intuitive version of the \cpp{link()} function template references the
target object directly via their group or field instance. The next example shows
the canonical case where a new link to the detector data is generated below the
data group. 
\begin{cppcode}
h5::nxfile  f          = h5::nxfile::open_file(data,true);
h5::nxgroup entry      = f.root()["entry"];
h5::nxgroup instrument = entry["instrument"];
h5::nxgroup detector   = instrument["detector"];
h5::nxgroup datagroup  = entry["data"];
h5::nxfield det_data   = detector["data"];

link(det_data,data_group,"data");
\end{cppcode}
With this approach only internal links can be created as the target and the
parent object for the new link must be located in the same file. 
This approach is rather tedious as we have to open all the groups (for now we
have no better way - see the advanced section). However, one can exploit a
feature that \nexus-links have in common with file system links: the target
object must be available when the link is created. Consequently we can describe
the target also merely by its path
\begin{cppcode}
h5::nxgroup data_group = ...;

link("../instrument/detector/data",data_group,"data");
\end{cppcode}
Here the first argument (the target) is described by a string with the path to
the target. In this case the path is relative to the parent group of the link. 
Alternatively one can also use an instance of \cpp{nxpath} to reference the 
target object
\begin{cppcode}
h5::nxgroup data_group = ....;
nxpath target_path = nxpath::from_string("../instrument/detector/data");

link(target_path,data_group,"name");
\end{cppcode}
There is one important restriction a path has to obey when it should be used as
a reference to a link target: the path must not contain type only elements. 
For instance the path \cpp{"../:NXinstrument/:NXdetector/data"} would be a
perfect valid \nexus-path. However, when using such a path \cpp{link} will throw
a \cpp{value\_error} exception. The reason for this is simple: \libpniio\
currently uses the HDF5 linking mechanism which has no idea about \nexus-group
types!
Thus, as a rule of thumb, use only names within the path used to reference a
link target.

There is some other subtle issue with paths to link targets. They must not
comprise intermediate parent directory references (\cpp{..}). 
This cannot be handled by HDF5 correctly. \cpp{..} is only allowed at the 
beginning of a relative path. Absolute paths must not contain any \cpp{..} 
elements. 
Thus \cpp{../../entry/instrument} would be a valid target path for a link, 
but \cpp{/../entry/../entry} would not. 

%%%===========================================================================
\subsection{Create external links}

To create external links within a \nexus-file use the \cpp{link} function as for
internal links. However, only the string and path version of \cpp{link} can be
used to create external links. 
The only thing one has to do in order to obtain a link add the filename to the
path. Lets start again with the canonic example, the external detector file
\begin{cppcode}
h5::nxgroup detector_group = ....; //open the detector group in the master file

link("detector.nxs://entry/instrument/detector/data",detector_group,"data");
\end{cppcode}

%%%===========================================================================
\subsection{Link inquiry}

\libpniio\ (or better HDF5) distinguishes three kinds of links
\begin{inlinetab}{m{0.3\linewidth}m{0.5\linewidth}}
hard links & these are links to an object which are typically created 
             from the parent object to a newly created field or group \\
soft links & this is what one typically creates with the \cpp{link} function
template \\
external links &  are links that reference an object in a different file.
\end{inlinetab}
The major difference between hard- and soft-links is the fact that the latter
ones can dangle (the object they point to must not exist). 
\libpniio\ provides utility functions to check what kind of link one deals with. 
\begin{cppcode}
h5::nxgroup parent = ...;
string name = "data";

if(is_hard_link(parent,name))
    std::cout<<"hard link";
else if(is_soft_link(parent,name))
    std::cout<<"soft link";
else if(is_external_link(parent,name))
    std::cout<<"external link";
else
    std::cout<<"unkown link type";
\end{cppcode}
Each of the function takes the parent object of the link as its first and the
name of the link as its second argument. 

