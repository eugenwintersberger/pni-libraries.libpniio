%%%describing the basic usage

This chapter deals with the basic interface provided by the layer 1 types
implemented in \libpniio. All types concerning Nexus reside in one of the
namespaces embedded in {\tt pni::io::nx}. The namespaces below this one 
indicate either a particular storage backend (currently only HDF5 is
implemented).

To use the Nexus part of the library just add 
\begin{cppcode}
#include <pni/io/nx/nx.hpp>
\end{cppcode}
to your source file. 

%%%===========================================================================
\section{Working with files}
\input{tex/nexus_files.tex}


%%%===========================================================================
\section{Working with groups}

\nexus\ groups are instances of the \nxgroup\ template. They can be considered
as containers for fields and other groups and expose an STL compliant interface. 
To start working with groups in a file one hast to first obtain the root group 
with 
\begin{cppcode}
h5::nxfile file = h5::nxfile::open_file("test.nxs");
h5::nxgroup root = file.root();
\end{cppcode}

\subsection{Creating groups}

New groups are created by means of the \cpp{create\_group} member function of
\nxgroup
\begin{cppcode}
h5::nxgroup entry = root.create_group("scan_1","NXentry");
\end{cppcode}
This method takes two arguments where the first one is mandatory and denotes the
name of the group while the second one is optional and determines the
\nexus-class of the group. If the last argument is omitted a simple HDF5 group
is created (without an \cpp{NX\_class}  attribute).

Like files, groups are automatically destroyed when an instance looses scope,
but they can also be deliberately closed using their \cpp{close()} method.

%%%===========================================================================
\subsection{Accessing children}

Access to the direct children of a group instance is given via the 
\cpp{at()} method or the \cpp{[]} operator. Both accept either a numeric index 
of a child or its name as an argument. To loop over all children of the 
root group the following code could be used
\begin{cppcode}
h5::nxfile f = ....;
h5::nxgroup root = f.root();

for(size_t i=0;i<root.size();++i) std::cout<<root[i].name()<<std::endl;
\end{cppcode}
As for STL containers, the \cpp{size()} method returns the number of children 
of a group. To access a particular group via its name one can use
\begin{cppcode}
h5::nxfile f = ....;
h5::nxgroup root = f.root();

h5::nxgroup entry = root["entry"]; //alternatively root.at("entry");
\end{cppcode}
Unlike for STL containers both access variants (\cpp{at()} or \cpp{[]}) will 
throw an exception if a particular child could not be found or the index passed
exceeds the total number of children of the group. In addition to this simple 
access interface \nxgroup\ also exposes a fully STL compliant iterator 
interface. However, in order to use it some more deeper knowledge about 
\libpniio\ is required and thus this topic will be dealt with in
Section~\ref{section:group_iteration}.

%%%===========================================================================
\subsection{Other group related member functions}

Like files, groups posses an \cpp{is\_valid()} method which allows checking the 
state of a group. Similar to files, default constructed instances of \nxgroup\
are not valid. 
\begin{cppcode}
h5::nxgroup entry; 

if(!entry.is_valid()) std::cerr<<"The entry group is not valid!"<<std::endl;
\end{cppcode}
The getter methods \cpp{name()} and \cpp{filename()} return the name of the
group and the name of the file the group is stored in respectively.
Finally the \cpp{parent()} function returns the parent group of the a group.
In order to use the \cpp{parent()} member function a bit more extra care is 
used. When using the method in a simple way like 
\begin{cppcode}
h5::nxgroup p = other_group.parent();
\end{cppcode}
everything will be fine. However, when we want to use the return value of 
\cpp{parent()} as a temporary we have to do an explicit conversion to 
\cpp{nxgroup} like this
\begin{cppcode}
std::cout<<h5::nxgroup(entry_group.parent())<<std::endl;
\end{cppcode}
The reason for this is that \cpp{parent()} does not really return an 
instance of \cpp{nxgroup} but rather of \cpp{nxobject}. 
But \nxobject\ can be converted to \nxgroup\ safely. The reason 
for this behavior will be explained in detail in Section~\ref{section:nxobject}.




%%%===========================================================================
\section{Working with fields}

Fields are the data holding units in Nexus.  In \libpniio\ fields are 
implemented by the {\tt nxfield} template. Before diving into the details there
are some important general remarks which must be made. One can imagine a field as a
multidimensional array of virtually arbitrary data. It is impossible to create 
a scalar field with \libpniio. The reason for this is simple: every field should
be extendible if needed. :w


\subsection{Creating fields}

Field creation is rather complex as there are many options which should be taken
into account. Lets start with a simple example
\begin{cppcode}
h5::nxgroup entry = root["entry"];
h5::nxfield field = entry.create_field<float32>("temperature");
\end{cppcode}
This creates a 1D field

\subsection{Reading and writing data}

Fields provide two basic methods for reading and writing data: {\tt read()} and 
{\tt write()}. 
Both member functions accept a single argument which can be an instance of the
following types
\begin{center}
    \begin{tabular}{l|p{0.6\linewidth}}
        {\bf type} & {\bf description} \\
        \hline
        \hline
        {\tt mdarray<...>} & an instance of the {\tt mdarray} template \\
        \hline
        {\tt array} & an instance of the array type erasure \\
        \hline
        {\tt T\& } & a single scalar value of the fields element type or a 
        convertible type \\
        \hline
    \end{tabular}
\end{center}
There is an additional version of the {\tt read()} and {\tt write()} methods
with the signatures {\tt read(size\_t n,T *value)}
and {\tt write(size\_t n,T *value)}. These functions exist as interface to
legacy C code where only pointers are available. As the user has to provide the
number of elements in the memory region where the pointers point to these
functions try to add some additional safety.

A typical example would look like this
\begin{cppcode}
auto data = dynamic_array<uint32>::create(shape_t{1024,1024});
h5::nxfield background = ....;

//writing
background.write(data);

//reading
background.read(data);
\end{cppcode}

\subsection{Partial reading and writing}

Nexus fields support partial IO with the {\tt ()} operator which applies
selections to the field instances. 
\begin{cppcode}
h5::nxfield spectra = ...;

auto s = spectra.shape<shape_t>();

auto spectrum = dynamic_array<uint32>::create(shape_t{s[1]});

for(size_t i=0;i<s[0];++i)
{
    spectra(i,slice(0,s[1])).read(spectrum);
    .... process data ...
}
\end{cppcode}
The selection mechanism works the same as for the {\tt mdarray} template. 

\subsection{Field inquiry}

%%%===========================================================================
\section{Working with attributes}



